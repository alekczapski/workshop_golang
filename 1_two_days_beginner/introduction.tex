\documentclass[11pt, letterpaper]{article}
\usepackage[utf8]{inputenc}
\usepackage{hyperref}
\usepackage{enumerate}
\usepackage{listings}
\usepackage{fancyvrb}
\usepackage{kotex}
\usepackage{polski}
\usepackage[document]{ragged2e} % keep all left
\usepackage[english]{babel}

\usepackage{minted} % yaml syntax highlighting

\newenvironment{markdown}%
    {\VerbatimEnvironment\begin{VerbatimOut}{tmp.markdown}}%
    {\end{VerbatimOut}%
        \immediate\write18{pandoc tmp.markdown -t latex -o tmp.tex}%
        \input{tmp.tex}}

\providecommand{\tightlist}{%
  \setlength{\itemsep}{0pt}\setlength{\parskip}{0pt}}

\newcommand*{\email}[1]{\href{mailto:#1}{\nolinkurl{#1}} } 

\title{2 days Golang Programming Workshop\\CloudNative Week\\{ \small \href{https://creativecommons.org/licenses/by/4.0/}{CC BY 4.0} }  }
\author{Wojciech Barczynski\\(wbarczynski.pro@gmail.com)}
\date{}


\begin{document}
\selectlanguage{english}

\begin{titlepage}
\maketitle
\end{titlepage}

\tableofcontents
\pagebreak
\section{Prerequiments}

\subsection{Audience}

We design the workshop with the following assumptions about the audience:

\begin{itemize}%
\item Have 1-year experience in other programming language.%
\item Feel good with Command Line Interface.
\end{itemize}%

\subsection{Your workstation}

\begin{itemize}%
\item Linux or OSX recommended.%
\item Basic: \begin{itemize}%
    \item Golang
    \item a configured IDE or editor
    \item Git
    \end{itemize}%
\item Package manager exercise:
\begin{itemize}%
    \item godep
    \end{itemize}
\item {\small SQL} and no{\small SQL} exercise (recommended with docker):
\begin{itemize}%
    \item Postgres
    \item MongoDB
\end{itemize}
\item Nice to have: \begin{itemize}%
    \item Docker
    \end{itemize}
\end{itemize}

Check \href{https://github.com/golang/go/wiki/IDEsAndTextEditorPlugins}{Go Wiki} to see how to configure your favorite editor to write golang programs.

%%
\subsection{Verify the setup}
Let's run a hello world to check whether you can run go applications on your workstation.

\bigskip
\textbf{Notice:} No copy\&paste, please.

\bigskip
1. Let's check the (in)famous GOPATH:

\begin{verbatim}

$ printenv | grep GOPATH
GOPATH=/Users/wb/workspace2/goprojects
\end{verbatim}

\bigskip
2. Create a simple go program

\begin{verbatim}
$ mkdir -p $GOPATH/src/workshop-check
\end{verbatim}

\begin{verbatim}
$ cd $GOPATH/src/workshop-check
$ touch main.go
\end{verbatim}

\begin{verbatim}
$ cd $GOPATH/src/workshop-check
$ vim main.go
\end{verbatim}

The main.go should have the following text:

\begin{minted}[frame=single]{go}
package main

import "fmt"

func main() {
    // 1 tabulator
    fmt.Println("Hello! YOUR_NAME")
}
\end{minted}

Now, let's run it:

\begin{verbatim}
# 1. enforce formatting:
$ gofmt -w .

# 2. run
$ go run main.go

# 3. build and run:
$ go build .
$ ls
main.go  workshop-check

$ ./main.go
\end{verbatim}

Imagine for a second, we want to have it as a tool in our PATH: 

\begin{verbatim}
# install
$ go install

# magic?
$ ls $GOPATH/bin | grep workshop

$ export PATH=$PATH:$GOPATH/bin
$ cd

# run it
$ workshop-check
\end{verbatim}

%%%
\subsection{Golang Playground}

Open the browser and run our program on golang playground: https://play.golang.org/.

Notice: you can generate a link to your code sample.

%
\section{Basics}

\bigskip
\textbf{Notice:} No copy\&paste!
\bigskip

\subsection{Variable definition}
\bigskip
0. Create new directory in your \verb|$GOPATH|: \verb|hello-worlder|. Copy the \verb|main.go| from our \verb|workspace-check| project.

\bigskip
1. Please extract your name as a variable, use the following definitions and mark the incorrect ones:

\begin{minted}{go}
// 1
var myName string = "Natalia"
var myName = "Natalia"

// 2
myName := "Natalia"

// 3
var myName
myName = "Natalia"

// 4
var myName string
myName = "Natalia"

// 5
var myName string
myName := "Natalia"
\end{minted}

Notice: in Golang, we use camelCase for variable names.

\bigskip
2. Mark the myName as \verb|const|. 

\bigskip
3. Declare a variable for your home country and city, use the following construct:

\begin{minted}[frame=single]{go}
var (
  x = 10
  y = 20
)
\end{minted}

%%
\subsection{Integers and Floats}

No big surprise here, numbers:

\begin{itemize}
\item \verb|int|, \verb|int8, ..., int64|, byte $\rightarrow$ int8, rune $\rightarrow$ int32
\item \verb|unint|, \verb|uint8, ..., uint64|
\item \verb|float64|, \verb|float32, float64|
\item \verb|complex|
\end{itemize}

Golang does not support automatic conversion between types. Let's experience it.

\bigskip
1. Declare a variable \mintinline{go}{devExpDays} and \mintinline{go}{msg}:

\begin{minted}[frame=single]{go}
package main

import (
  "fmt"
  "reflect"
  "strconv"
)

func main() {
  name := "Natalia"
  devExpDays := 365
  msg := name + " has " + devExpDays + " exp as developer"
  fmt.Println(msg)
}
\end{minted}

Run it. What error message did you see?

\bigskip
2. To make it running, we need to use \mintinline{go}{strconv.Itoa}. Add the following import and call the function:

\begin{minted}[frame=single]{go}
import (
  "strconv"
  "fmt"
)
\end{minted}

\bigskip
3. Now let's go back from string to integer:

\begin{minted}{go}
// imagine, we got it from the user:
devExpYears := "2"
devExpDays := 365 * devExpYears
\end{minted}

to convert \mintinline{go}{devExpYears} use the following code:

\begin{minted}[frame=single]{go}
// the famous error-return
days, err := strconv.Atoi("12020")
if err != nil {
  fmt.Printf("Cannot convert %v", err)
  return
}
\end{minted}

You have more functions to convert from basic types to string and back, check \href{https://golang.org/pkg/strconv}{Package strconv documentation}.

Notice: \mintinline{go}{import (_ "strconv")}.

%%%%%%%%%%%%%%%%%%%%%%%%%%%%%%%%%%%%
\subsection{Boolean}

Just to note: \mintinline{go}{true} and \mintinline{go}{false}, standard logical operators: \mintinline{go}{&&}, \mintinline{go}{!}, and \mintinline{go}{||}.

%%%%%%%%%%%%%%%%%%%%%%%%%%%%%%%%%%%%
\subsection{Math}
Nothing dramatic here. For more advance mathematical functions, you should check the \href{https://golang.org/pkg/math}{Package math}:
\begin{minted}{go}
import (
  "math"
)
\end{minted}

%%%%%%%%%%%%%%%%%%%%%%%%%%%%%%%%%%%%
\subsection{Slices and hidden arrays}
In golang, we use \verb|slices|, seldom we use arrays.

\bigskip
1. If you want to defined an array, you specify the length explicitly:

\begin{minted}[frame=single]{go}
arr1 := [...]string{"pa", "rr", "ot"}
arr2 := [3]string{"pa", "rr", "ot"}

fmt.Print(arr1)
fmt.Printf("%v", arr1)
\end{minted}

Slice, an interface of the array, on the other hand we create with:

\begin{minted}[frame=single]{go}
arr1 := [...]string{"pa", "rr", "ot"}

slice1 := []string{"pa", "na", "ma"}
slice2 := arr1[:]
\end{minted}

\bigskip
2. What is the output?

\begin{minted}[frame=single]{go}
  var three [3]int
  two := [2]int{10, 20}

  three = two

  fmt.Println(three)
  fmt.Println(two)
\end{minted}


\bigskip
3. Let's define our hello world messages and add one more:

\begin{minted}[frame=single]{go}
helloWorld := []string{"dzień dobry", "Hallo", "guten Tag"}
fmt.Printf("A: len: %d cap: %d \n", len(helloWorld),
  cap(helloWorld))

czechia := "Ahoj"
helloWorld = append(helloWorld, czechia)

fmt.Printf("B: len: %d cap: %d \n", len(helloWorld),
  cap(helloWorld))

fmt.Printf("%v\n", helloWorld)
\end{minted}

Note down the len and cap in \verb|A| and \verb|B|:

\bigskip
\bigskip
\bigskip
4. Slices of slices. How would you write a one liner to print out:

\begin{itemize}
  \item \mintinline{go}{["dzień dobry"]}: \bigskip
  \item middle hallo messages: \bigskip
  \item all except the last one: \bigskip
  \item just 1rst element: \bigskip
  \item just 15th element: \bigskip
\end{itemize}

Hint: use \verb|slice[x]|, \verb|slice[x:]|, \verb|slice[:x]|, and \verb|slice[x:y]|

\bigskip
5. Watch out, the slices might bite your head off. Note, slice has \emph{capacity}, \emph{length}, and \textbf{pointer to the underlaying array} (see \href{reflect.SliceHeader}{https://golang.org/pkg/reflect/\#SliceHeader}):
\begin{minted}[frame=single]{go}
package main

import "fmt"

func main() {
  helloWorld := []string{"dzień dobry", "Ahoj", "Goodmorning"}
  eastEuropeHello := helloWorld[0:2]
  fmt.Printf("len: %d cap: %d \n", len(eastEuropeHello),
    cap(eastEuropeHello))

  eastEuropeHello[0] = "Dobry Wieczor"
  fmt.Printf("%v\n", helloWorld)
  fmt.Printf("%v\n", eastEuropeHello)
}
\end{minted}
What is the result?

\bigskip
\bigskip
Replace \mintinline{go}{eastEuropeHello[0] = "Dobry Wieczor"} by:
\begin{minted}[frame=single]{go}
 eastEuropeHello = append(eastEuropeHello, "Dobry Wieczor")
 eastEuropeHello = append(eastEuropeHello, "Dobry Wieczor")
\end{minted}

What is the result? What has happend?

\bigskip
\bigskip
\bigskip
It should be not a suprise that: 

\begin{minted}[frame=single]{go}
  a := []int{1,3,5,7}
  b := []int{2, 4, 6}

  a = b

  a[0] = 99

  // prints the same
  fmt.Printf("%+v\n", a)
  fmt.Printf("%+v\n", b)
\end{minted}

\bigskip
6. Let's fix that with the following code snipped:

\begin{minted}[frame=single]{go}
newSlice := make([]string, 2)
copy(newSlice, slice)
\end{minted}

\bigskip
7. We can also use make to create a slice with desired length and capacity:

\begin{minted}[frame=single]{go}
  msgs := make([]string, 2, 20)
  msgs[0] = "Ahoj"
  msgs[1] = "Goodmorning"

  for idx := range msgs {
    fmt.Printf("%s\n", msgs[idx])
  }

  for idx, v := range msgs {
    fmt.Printf("%d, %s\n", idx, v)
  }

  for _, v := range msgs {
    fmt.Printf("%s\n", v)
  }
\end{minted}

\bigskip
8. Note, when a function returns no result, use a \emph{nil slice}:\\ \mintinline{go}{var nilSlice []string = nil}.

%%%%%%%%%%%%%%%%%%%%%%%%%%%%%%%%%%%%
\subsection{Control structure: Loops}

In go there is only one loop keyword.

\begin{minted}[frame=single]{go}
  for i := 0; i < 10; i++ {
    fmt.Println(i)
  }

  for i < 10 {
    fmt.Println(i)
    i++
  }
  // also map
  for index, value := range someSlice {
    fmt.Println(index, value)
  }
\end{minted}

%%%%%%%%%%%%%%%%%%%%%%%%%%%%%%%%%%%%
\subsection{String}

String is a read-only slice of bytes. Go source code is always in UTF-8.

\bigskip
1. Run the following code, note down the results:

\begin{minted}[frame=single]{go}
const adress := "ul. Przeskok 2"

fmt.Printf("len: %s\n", len(adress))

fmt.Printf("1: %s\n", adress[0:3])
fmt.Printf("2: %c\n", adress[2])
fmt.Printf("3: %s\n", adress[2:])
fmt.Printf("4: %s\n", adress[5:])

fmt.Printf("5: %s\n", adress[16:])
fmt.Printf("6: %s\n", adress[:16])
\end{minted}


\bigskip
2. Use the following example to build your own program printing out your~3~favorite emoicons:

\begin{minted}[frame=single]{go}
const milk = "우유"

for index, runeValue := range milk {
  fmt.Printf("%c (%U) starts at byte position %d\n",
    runeValue, runeValue, index)
}
\end{minted}

\bigskip
3. What does now happen?:

\begin{minted}[frame=single]{go}
const milk = "우유"

  for i := 0; i < len(milk); i++ {
    fmt.Printf("byte: %x at the index %d\n",
      milk[i], i)
  }
\end{minted}

\bigskip
Notice: Important packages are \verb|Package strings| and \verb|Package unicode/utf8|.

%%%%%%%%%%%%%%%%%%%%%%%%%%%%%%%%%%%%
\subsection{Maps}

Build a program that displays a hello-world message for different languages.

1. Define the map:

\begin{minted}[frame=single]{go}
  helloMsgs := map[string]string{}
  helloMsgs["pl"] = "Dzień Dobry"
  helloMsgs["en"] = "Good morning" 
\end{minted}

\bigskip
2. Read input from the user:

\begin{minted}[frame=single]{go}
  helloMsgs := map[string]string{}
  helloMsgs["pl"] = "Dzień Dobry"
  helloMsgs["en"] = "Good morning"

  var lang string
  // read a single world in ASCII
  // skip error handling
  fmt.Scan(&lang)
\end{minted}

\bigskip
2. Print the hello message:

\begin{minted}[frame=single]{go}
if val, ok := helloMsgs[lang]; ok {
  // found
} else {
  // not found
}
\end{minted}

\bigskip
3. Notice, golang has a cool feature:

\begin{minted}[frame=single]{go}
  helloMsgs := map[string]string{
    "pl": "Dzień Dobry",
    "en": "Good morning",
  }
\end{minted}

We want to have different greetings depending on the time of day:

\begin{minted}[frame=single]{go}
  helloMsgs := map[string]map[string]string{
    "pl": {"morning": "Dzień Dobry"},
    "en": {"morning": "Good morning"},
  }
\end{minted}

Now, users got bored with the same greeting:

\begin{minted}[frame=single]{go}

  helloMsgs := map[string]map[string][]string{
    "pl": {"morning": []string{
      "Dzień Dobry",
      "Piękny poranek",
    }},
    "en": {"morning": []string{
      "Good morning",
      "Morning",
    }},
  }
\end{minted}

Let's make it more readable:

\begin{minted}[frame=single]{go}
  type daytimeGreetings map[string][]string
  type g map[string]daytimeGreetings

  helloMsgs := map[string]daytimeGreetings{
    "pl": {"morning": []string{
      "Dzień Dobry",
    }},
    "en": {"morning": []string{
      "Good morning",
    }},
  }
\end{minted}

You can use this declarative style to define even the most complex JSON structures.

\bigskip
4. Write a program that randomize the messages, use \emph{math/rand} and \emph{time} packages to initialize random seed.

\bigskip
5. [Homework] Use time information to find out which part of the day we have.

%%%%%%%%%%%%%%%%%%%%%%%%%%%%%%%%%%%%
\subsection{User defined type}

A short exercise to show you the types you can define over the basic and~composite types:

%%%%%%%%%%%%%%%%%%%%%%%%%%%%%%%%%%%%
\subsection{Type Definitions}

\begin{minted}[frame=single]{go}
package main

import "fmt"

type myInt int

func display(i int) {
  fmt.Printf("%d", i)
}

func main() {
  var i myInt = 12
  i = i + 12
  // how to fix it?
  display(i)
}
\end{minted}

%%%%%%%%%%%%%%%%%%%%%%%%%%%%%%%%%%%%
\subsection{Type Alias Declarations}

\begin{minted}[frame=single]{go}
package main

import "fmt"

type myInt = int

func display(i int) {
  fmt.Printf("%d", i)
}

func main() {
  var i myInt = 12
  i = i + 12
  display(i)
\end{minted}

%%%%%%%%%%%%%%%%%%%%%%%%%%%%%%%%%%%%
\subsection{fmt.Printf}

Check the \href{golangbyexample}{https://gobyexample.com/string-formatting}.

%%%%%%%%%%%%%%%%%%%%%%%%%%%%%%%%%%%%
\subsection{Functions}

Let's split our program into nice functions:

\bigskip
1. Let's move the logic for displaying the hello message to a separate function:

\begin{minted}[frame=single]{go}
  helloMsgs := map[string]string{
    "pl": "Dzień Dobry",
    "en": "Good morning",
  }
\end{minted}

The function should take as arguments the map with messages and language.

\bigskip
2. Let's add return \verb|true| if we have a message for given language and \verb|false| if not. Notice:

\begin{minted}[frame=single]{go}
func displayHello() (found bool) {
  // notice empty return
  return
}
\end{minted}

\bigskip
2. Let's follow the golang way and return an error:

\begin{minted}[frame=single]{go}
func displayHello() (err error) {
  err = fmt.Errorf("Unsupported language")
  return
}

func main() {
  err := displayHello()
  if err != nil {
    fmt.Printf("Not found!!! %v", err)
    return
  }

}
\end{minted}

\bigskip
3. Functions are first class citizens in Golang and we often use them as arguments:

\begin{minted}[frame=single]{go}
type letterDecorator func(string) string

//func printLetters(msg []string, decorator func(string) string)
func printLetters(msg []string, decorator letterDecorator) {
  for _, l := range msg {
    d := decorator(l)
    fmt.Println(d)
  }
}

func main() {
  alphabet := []string{"a", "b", "c", "d", "d"}

  printLetters(alphabet, func(s string) string {
    return strings.ToUpper(s)
  })

  printLetters(alphabet, func(s string) string {
    return "::" + strings.ToLower(s) + "::"
  })
}
\end{minted}

\bigskip
4. Notice: we have support for variadic parameters in functions:\\ \mintinline{go}{func printSymbols(msg ...string)}:

\begin{minted}[frame=single]{go}
printSymbols()
printSymbols("a", "z")
printSymbols(alphabet...)
\end{minted}

\bigskip
5. We can move the execution of a function to the end of the scope with \verb|defer|:

\begin{minted}[frame=single]{go}
package main

import "fmt"
  
func main() {
  defer fmt.Println("booom!")

  for i := 0; i < 10; i++ {
     fmt.Println("tick...")
  }
}
\end{minted}

%%%%%%%%%%%%%%%
\subsection{Control structure: if and switch}

\subsubsection{switch}

The \emph{switch} can work on any data type, you do not need to switch on a value, see: \href{Golang Wiki}{https://github.com/golang/go/wiki/Switch}.

The very common case for \emph{switch} is to check types:

\begin{minted}[frame=single]{go}
func do(v interface{}) string {
  switch u := v.(type) {
  case int:
    return strconv.Itoa(u*2) // u has type int
  case string:
    mid := len(u) / 2 // split - u has type string
    return u[mid:] + u[:mid] // join
  }
  return "unknown"
}
\end{minted}

\pagebreak
\subsubsection{if}

The \emph{If} and \emph{else} works as in other programming languages, except that you can put a language expression:

\begin{minted}[frame=single]{go}
if err := dec.Decode(&val); err != nil {
  // handling error
}
// happy path
\end{minted}

Example from the \href{https://golang.org/src/net/net.go}{Package net}:

\begin{minted}[frame=single]{go}
if nerr, ok := err.(net.Error); ok && nerr.Temporary() {
  // here nerr is an instance of net.Error
  // and the error is Temporary
}
\end{minted}

Let's build an app for writing and reading a file:

\begin{minted}[frame=single]{go}
package main

import (
  "io/ioutil"
  "fmt"
)

func main() {
  // change to /dat1
  fPath := "/tmp/dat1"
  inD := []byte("hello\nWorld\n")
  if err := ioutil.WriteFile(fPath, inD, 0644); err != nil {
    panic(err) // failfast
  }
  fmt.Println("Write was successful!")

  outData, err := ioutil.ReadFile(fPath)
  if err != err {
     panic(err) // failfast
  }
  fmt.Print(string(outData))
}
\end{minted}

If it works, change \mintinline{go}{fPath} to \mintinline{go}{fPath := "/tmp/dat1"}. What does happen?

%%%%%%%%%%%%%%%
\subsection{Pointers}

Remember:

\begin{itemize}
\item Because of the Golang design, your code will work usually faster if you pass small data types by value.
\item Do not be overzealous with the pointers.
\item Maps, slices, and pointers are reference types.
\end{itemize}

Write the following program:

\begin{minted}[frame=single]{go}
package main

import "fmt"

func tryAnswerEverything(i int) {
  i = 45
}

func answerEverything(i *int) {
  *i = 42
}

func main() {
  i := 33

  fmt.Println(i)
  tryAnswerEverything(i)
  fmt.Println(i)

  answerEverything(&i)
  fmt.Println(i)
}
\end{minted}

Notice, we can return in functions pointers to local variables: \mintinline{go}{func Answer() *int}.

%%%%%%%%%%%%%%%%%%%%%%%%%%%%%%
\subsection{Structures}

In Golang, we do not have \emph{classes}, instead \emph{struct} with \emph{methods}. Our language does not support inheritance, favors composition instead. 

\subsection{Structures and Methods}

1. Write a program to manage employees:

\begin{minted}[frame=single]{go}
type Employee struct {  
    FirstName   string
    LastName    string
    leavesTotal int
    LeavesTaken int
}

func (empl *Employee) TakeHolidays(days int) error {
  // write an implementation
} 

func (empl *Employee) limitExceeded(days int) bool {
  // write an implementation
}

func main() {
  empl := new(Employee)
  empl.FirstName = "Laste"
  empl.LastName = "BB"
  empl.leavesTotal = 26
  fmt.Println(empl.FirstName)
  // ...
}
\end{minted}

\bigskip

Question? How to init the Employee outside our package?

\begin{minted}[frame=single]{go}
// hide the implementation
type employee struct {  
}

// provide a factory method New or NewEmployee
func NewEmployee(firstName string, lastName string,
    leaveDays int) *employee {
  
  return &employee{FirstName: firstName, LastName: lastName,
    totalLeaves : leaveDays}
}

// called: employee.New() where employee is a package name 
\end{minted}

\bigskip

2. Refactor our application and move implementation of the \verb|employee| to a separate package. To do it, create a directory \verb|employee| and create inside \verb|employee.go|:

\begin{minted}[frame=single]{go}
package employee

type Employee struct {
  FirstName   string
  LastName    string
  leavesTotal int
}

func New(firstName string, lastName string,
    leaveDays int) *Employee {

  return &Employee{FirstName: firstName, LastName: lastName,
    leavesTotal: 30}
}
\end{minted}

Create an instance of Employee in \verb|main.go|.

%%%%%%%%%%%%%%%%%%%%%%%%%%%%%%
\subsection{Pointer receiver vs value receiver}

What is the difference?

\begin{minted}[frame=single]{go}
func (empl *Employee) TakeHollidays(taken int) {
    empl.leavesTaken = empl.leavesTaken + taken
}
\end{minted}

and:

\begin{minted}[frame=single]{go}
func (empl Employee) TakeHollidays(taken int) {
    empl.leavesTaken = empl.leavesTaken + taken
}
\end{minted}

Write a program to find out.

%%%%%%%%%%%%%%%%%%%%%%%%%%%%%%
\subsection{Structures and Interfaces}

Your structure has to implement the functions from the interface:

\begin{minted}[frame=single]{go}
package main

import "workshop-app/postgres"

type DataStore interface {
  GetEmployee(id int) string
}

type App struct {
  ds DataStore
}

func main() {
  app1 := App{ds: &postgres.PsqlStore{}}
  fmt.Print(app1.ds.GetEmployee(12))
}
\end{minted}

We might have two configurable stores, one psql:

\begin{minted}[frame=single]{go}
package postgres

type PsqlStore struct {
}

func (ps *PsqlStore) GetEmployee(id int) string {
  return "psql"
}
\end{minted}

and one, mongodb:

\begin{minted}[frame=single]{go}
package mongo

type MongoStore struct {
}

func (ps *MongoStore) GetEmployee(id int) string {
  return "mongo"
}
\end{minted}

Run the app and verify that the mongoStore works as well. When it works, break it, for e.g., change the type of args in \mintinline{go}{MongoStore}.

%%%%%%%%%%%%%%%%%%%%%%%%%%%%%%
\subsection{Composition instead of Inheritance}

\begin{minted}[frame=single]{go}
type person struct {
  firstName string
}

func (p person) name() string {
  return p.firstName
}

type employee struct {
  EmployeeID string
  person
}

func main() {
  empl := employee{}
  empl.firstName = "Wojtek"
  empl.person.firstName = "Krzyś"
  fmt.Println(empl.firstName)
}
\end{minted}

Notice: For polymorphism, you need to use \emph{interface}.

%%%
\subsection{Interfaces vs Functions}

If you get high on interface, do not forget about the functions, an example from~\href{https://golang.org/src/net/http/server.go}{net/http}:

\begin{minted}[frame=single]{go}
type HandlerFunc func(ResponseWriter, *Request)

// ServeHTTP calls f(w, r).
func (f HandlerFunc) ServeHTTP(w ResponseWriter, r *Request) {
    f(w, r)
}
\end{minted}

we could use it to customize our employees leave policies:

\begin{minted}[frame=single]{go}
package employee

type LeaveHandler func(taken int, total int) bool

func NewLeaveHandler(f func(int, int) bool) LeaveHandler {
  return f
}

func (f LeaveHandler) CanTakeHollidays(empl Employee) bool {
  r := f(empl.totalLeaves, empl.leavesTaken)
  return r
}
\end{minted}

%%%%%%%%%%%%%%%%%%%%%%%%%%%%%%%%%%%%%%%%%%%%%%%%%%%%%%%%%%%%%%%%
\subsection{Packages, go get, godep, go mod}

The traditional way to get your packages was \verb|go get|. At this moment, we have two approaches \verb|v.go| (new, favored by Google) and \verb|godep| (old, community driven). Let's see how they work.

\bigskip
1. Create new project \emph{workshop-gomod}, with main.go:

\begin{minted}[frame=single]{go}
package main

import (
  "fmt"
  "net/http"

  "github.com/gorilla/mux"
)

func HelloHandler(w http.ResponseWriter, r *http.Request) {
    w.WriteHeader(http.StatusOK)
    fmt.Fprintf(w, "Hello World")
}

func main() {
    r := mux.NewRouter()
    r.HandleFunc("/", HelloHandler)
    http.Handle("/", r)
}
\end{minted}


\bigskip
2. Use \emph{gomod} to capture the deps:

\begin{minted}{text}
$ GO111MODULE=on go mod init
$ GO111MODULE=on go get github.com/gorilla/mux
$ ls

go.mod  go.sum  main.go

$ GO111MODULE=on go mod vendor
$ ls

$ ls vendor

$ GO111MODULE=on go list -m all
\end{minted}

Notice: the debate is ongoing whether we should or must not commit \verb|vendor| to your repo. You should, at least, push \verb|go.mod| and \verb|go.sum| to your repo.

\bigskip

3. Create new project \emph{workshop-godep}, with main.go:

\begin{minted}[frame=single]{go}
package main

import (
  "fmt"
  "net/http"

  "github.com/gorilla/mux"
)

func HelloHandler(w http.ResponseWriter, r *http.Request) {
    w.WriteHeader(http.StatusOK)
    fmt.Fprintf(w, "Hello World")
}

func main() {
    r := mux.NewRouter()
    r.HandleFunc("/", HelloHandler)
    http.Handle("/", r)
}
\end{minted}

\bigskip

4. Use \emph{godep} to manage your dependencies:

\begin{minted}{text}
$ dep
$ dep init -v
$ ls

vendor  Gopkg.lock  Gopkg.toml  main.go
\end{minted}

You will have less surprises with \verb|dep| at this moment. \verb|v.go| will be the default soon, consult \href{golang wiki}{https://github.com/golang/go/wiki/Modules}.

\subsection{Linters}

Linters are a part of the language:

\begin{markdown}
- gofmt
- goimport - gofmt + sorting imports
- govet - now executed with tests
\end{markdown}

To apply the fixes, use \verb|-w| flag:

\begin{minted}{text}
$ gofmt -w .
$ goimports -w *.go && goimports -w */*.go
\end{minted}

There are many linters out there, check, for e.g., \href{https://github.com/golangci/awesome-go-linters}{awesome-go-linter page}.

You should call the linters as part of your CI/CD pipeline:

\begin{markdown}
- linter
- test
- integration-test
\end{markdown}

%%%%%%%%%%%%%%%%%%
\subsection{Errors}
Let's define our own error and see how they work. We will use the standard libraries for handling errors.

Notice \verb|error| interface is:

\begin{minted}[frame=single]{go}
type error interface {
  Error() string
}
\end{minted}

%%%
\subsubsection{Sentimental Errors}

An example of such errors are: \mintinline{go}{sql.ErrNoRows} and \mintinline{go}{io.EOF}, they are declared as:

\begin{minted}[frame=single]{go}
package sql

var errNoRows = errors.New("sql: no Rows available")
\end{minted}

The advantage? It is simple to handle:

\begin{minted}[frame=single]{go}
err := db.QueryRow("SELECT * FROM users WHERE id = ?", userID)
if err == sql.ErrNoRows {
  // an error we know
}  else if err != nil { 
  // another eeror
}
\end{minted}

Disadvantage? Not too much info, they become a part of your package API.

\bigskip

Please implement, an error for our Employee application using \emph{Package errors}\footnote{\href{https://godoc.org/github.com/pkg/errors}{https://godoc.org/github.com/pkg/errors}}.

%%%
\subsubsection{Custom Error Types}

Use the following example of a custom error type to change the errors implementation in the \verb|Employee| application:

\begin{minted}[frame=single]{go}
type ParsingError struct {
    IncorrectValue string
}

func (e *ParsingError) Error() string {
    return fmt.Sprintf("Cannot parse %v: internal error",
      e.IncorrectValue)
}
\end{minted}

if we use our custom error code, our code will get more complicated:

\begin{minted}[frame=single]{go}
if err := Foo(); err != nil {
    switch e := err.(type) {
    case *ParsingError:
        // 
    default:
        log.Println(e)
    }
}
\end{minted}

\bigskip

\subsubsection{Opaque errors}

The idea:

\begin{itemize}
\item return your own errors
\item provide functions to determine what has happended
\end{itemize}

An example for Dave Cheney blog \footnote{\href{https://dave.cheney.net/2016/04/27/dont-just-check-errors-handle-them-gracefully}{https://dave.cheney.net/2016/04/27/dont-just-check-errors-handle-them-gracefully}}:

\begin{minted}[frame=single]{go}
type temporary interface {
        Temporary() bool
}
 
// IsTemporary returns true if err is temporary.
func IsTemporary(err error) bool {
        te, ok := err.(temporary)
        return ok && te.Temporary()
}
\end{minted}

%%%
\subsubsection{Wrapping/Annotating and Unwrapping errors}

With help of the package errors\footnote{\href{https://godoc.org/github.com/pkg/errors}{https://godoc.org/github.com/pkg/errors}}, we can provide support for the stacktraces and handling error causes.

Let's see how to implement, a stacktrace support for our application: 

\begin{minted}[frame=single]{go}
package main

import (
  "fmt"
  "github.com/pkg/errors"
)

type stackTracer interface {
  StackTrace() errors.StackTrace
}

var errProcess = errors.New("boom")

func processData() error {
  return errProcess
}

func main() {
  err := processData()
  wrappedErr := errors.Wrap(err, "processing failed")
  fmt.Printf("%v", wrappedErr)

  if err, ok := wrappedErr.(stackTracer); ok {
    fmt.Printf("%+v", err.StackTrace())
  }
}
\end{minted}

Unwrapping:

\begin{minted}[frame=single]{go}
package main

import (
  "fmt"
  "net"

  "github.com/pkg/errors"
)

type myErrProcess error

var errProcess myErrProcess = errors.New("boom")

func processData() error {
  return errProcess
}

func main() {
  errData := processData()
  wrappedErr := errors.Wrap(errData, "processing failed")

  _, ok := errors.Cause(wrappedErr).(net.Error)
  if ok {
    fmt.Printf("net.Error")
  }

  errP, ok := errors.Cause(wrappedErr).(myErrProcess)
  if ok {
    fmt.Printf(errP.Error())
  }
}
\end{minted}

%%%%%%%%
\subsubsection{Defer, Panic, and Recover}

There is a way to recover from the panic, when we use \verb|defer|, \verb|panic|, and \verb|recover|. We are not going to cover it in the introduction course.

If you cannot wait, check \href{https://blog.golang.org/defer-panic-and-recover}{a defer-panic-and-recover blog post on golang.org} and \href{https://github.com/golang/go/wiki/PanicAndRecover}{the golang wiki PanicAndRecover article}.

\subsubsection{Error best practices}

\begin{itemize}
\item Handle errors once
\item Error, keep on the left
\item Notice: standard errors do not come with stacktraces
\item Check a good article from 8thlight \footnote{\href{https://8thlight.com/blog/kyle-krull/2018/08/13/exploring-error-handling-patterns-in-go.html}{https://8thlight.com/blog/kyle-krull/2018/08/13/exploring-error-handling-patterns-in-go.html}}
\end{itemize}

%%%%%%%%%%%%%%%%%%
\pagebreak
\subsection{Tests}

Create a simple test in a file -- \verb|main_test.go|. To run tests: \verb|go test .|.

\begin{minted}[frame=single]{go}
package main

import (
  "fmt"
)

func add(a int, b int) int {
  return a + b
}

func main() {
  fmt.Println(add(10,20))
}

func TestAdd(t *testing.T) {
  if add(10,25) != 20 {
    t.Fatal("Boom!")
  }
}
\end{minted}

\pagebreak
\subsubsection{Table-driven tests}

1. Create a project \verb|workshop-test|:

main.go:

\begin{minted}[frame=single]{go}
package main

import (
  "errors"
  "fmt"
)

var errUnknownOperation = errors.New("Unknown operation")

func Calculate(op string, a int, b int) (int, error) {
  switch op {
  case "+":
    return a + b, nil
  }
  return 0, errUnknownOperation
}

func main() {
  r, _ := Calculate("+", 1, 2)
  fmt.Println(r)
}
\end{minted}

main\_test.go:

\begin{minted}[frame=single]{go}
package main

import (
  "testing"
)

func TestCalculate(t *testing.T) {
  testCases := map[string]struct {
    op       string
    a        int
    b        int
    expected int
  }{
    "simple add": {"+", 1, 3, 4},
  }

  for name, v := range testCases {
    t.Logf("test: %s", name)
    r, err := Calculate(v.op, v.a, v.b)
    if err != nil {
      t.Fatalf("%v", err)
    }
    if r != v.expected {
      t.Fatalf("Failed!")
    }
  }
}
\end{minted}

Run the tests:

\begin{minted}{text}
$ go test .
\end{minted}

Notice: you can add \verb|-race| to turn on the race detector.

\bigskip
2. Add, first the test, support for division.

\subsubsection{Test with real X}

Golang developers prefer to work against real databases, file systems, etc.

\subsubsection{Tests short and long}

\begin{minted}[frame=single]{go}
if testing.Short() {
  t.Skip("skipping test in short mode.")
}
\end{minted}

\subsubsection{Integration tests}

The best practice is to use build tags to distinguish integration tests:

\begin{minted}[frame=single]{go}
// +build integration

package service_test

func TestSomething(t *testing.T) {
  if service.IsMeaningful() != 42 {
    t.Errorf("oh no!")
  }
}
\end{minted}

To run:

\begin{minted}{text}
$ go test --tags integration ./...
\end{minted}

\subsubsection{Look ahead}
There is much more:

\begin{itemize}
\item If your functions \href{https://blog.chewxy.com/2018/03/18/golang-interfaces/}{accept interfaces, return structs}, they are easier to test.
\item Check the brilliant blog on \href{https://peter.bourgon.org/go-for-industrial-programming/}{Go for Industrial Programming} and \href{https://www.youtube.com/watch?v=PTE4VJIdHPg}{the corresponding video}.
\item If you like the BDD style, look into \href{https://github.com/onsi/ginkgo}{ginko} and \href{gomega}{https://github.com/onsi/gomega}.
\end{itemize}

%%%%%%%%%%%%%%%%%%%%%%%%%%%%%%%%%
\pagebreak
\section{Your basic web app}

Knowing the basics of Golang, let's build a web application.

\subsection{Simplest}

Writing a web server in Golang, thanks to very solid standard library, is faily simple:

\begin{minted}[frame=single]{go}
package main

import (
  "io"
  "log"
  "net/http"
)

func main() {
  hello := func(w http.ResponseWriter, r *http.Request) {
    io.WriteString(w, "Hello World!")
  }

  // Run http server on port 8080
  err := http.ListenAndServe(":8080", http.HandlerFunc(hello))

  // Log and die, in case something go wrong
  log.Fatal(err)
}
\end{minted}

%%%%%%%%%%%%%%%%%%%%%%%%%
\subsection{Multiplexed}

To multiplex, we need to create a Multiplexer:

\begin{minted}[frame=single]{go}
package main

import (
    "io"
    "log"
    "net/http"
)

func main() {
  mux := http.NewServeMux()

  mux.HandleFunc("/hello", func(w http.ResponseWriter,
      r *http.Request) {
    io.WriteString(w, "Hello")
  })
  
  mux.HandleFunc("/world", func(w http.ResponseWriter,
      r *http.Request) {
    io.WriteString(w, "World")
  })

  log.Fatal(http.ListenAndServe(":8080", mux))
}
\end{minted}


%%%%%%%%%%%%%%%%%%%%%%%%%
\subsection{Handler as a struct}

To customize handler, we can create a \verb|struct|

\begin{minted}[frame=single]{go}
type MyHandler struct {
  Greeting string
}

func (h *MyHandler) ServeHTTP(w http.ResponseWriter,
    r *http.Request) {
  fmt.Fprintf(w, "%s, %s!", h.Greeting, r.RemoteAddr)
}

func main() {
  log.Fatal(http.ListenAndServe(":8080", &MyHandler{
    Greeting: "Hello World!",
  }))
}
\end{minted}

\subsubsection{Sharing data structures among handlers}

The following example shows how to share data among handlers, e.g., database connection details, configs:

\begin{minted}[frame=single]{go}
package main

import (
  "fmt"
  "log"
  "net/http"
)

type App struct {
  ServiceName string
  // Datasource
  // logging config
}

func (app *App) HelloWorld(w http.ResponseWriter,
    r *http.Request) {
  w.WriteHeader(http.StatusOK)
  fmt.Fprintf(w, "Hello World from " + app.ServiceName)
}

func main() {
  app := App{ServiceName: "MyApp"}
  mux := http.NewServeMux()
  mux.HandleFunc("/", app.HelloWorld)

  log.Fatal(http.ListenAndServe(":8080", mux))
}
\end{minted}

\subsubsection{Reading body}

Extend the previous example to read the data passed with http body:

\begin{minted}[frame=single]{go}
func (h *Handler) ServeHTTP(w http.ResponseWriter,
      r *http.Request) {
  var data bytes.Buffer // []byte with IO
  
  // body, err := ioutil.ReadAll(r.Body)
  n, err := data.ReadFrom(r.Body) // read body to the buffer
  if err != nil {  
    panic(err) 
  }

  log.Printf("Got %d bytes from %s: %s\n", n, r.RemoteAddr,
    data.String())
}
\end{minted}

Test it:

\begin{minted}{text}
$ curl -d '{"name": "natalia"}' 127.0.0.1:8080 
\end{minted}


\subsubsection{Parse URL}

We have also support for parsing URL in \href{https://golang.org/pkg/net/url/}{net/url Package}:

\begin{minted}[frame=single]{go}
// "lang=pl"
q := r.URL.Query()
lang := q.Get("lang")
\end{minted}

\subsubsection{Write multilingual hello-world app}

Multilingual hello-world app supports 

\begin{itemize}
\item US1: \emph{lang} on path \verb|/| as a GET parameter to specify language \\
       \emph{user} as a GET parameter to specify username for the greetings.
\item US2: \emph{lang} and  \emph{user} in body: \verb|user:Wojtek,lang:pl|.
\item if \emph{lang} is missing, return 400.
\item if \emph{user} is missing, return 404.
\end{itemize}

\subsubsection{Testing handlers}

Create tests to cover the edge cases:

\begin{minted}[frame=single]{go}
func TestHandlers(t *testing.T) {
  // Your handler to test
  handler := func(w http.ResponseWriter, r *http.Request) {
    http.Error(w, "Uh huh", http.StatusBadRequest)
  }

  // Create a request
  r, err := http.NewRequest("GET",
    "http://test.com?lang=pl&user=wojtek", nil)

  // Handle request and store result in w
  w := httptest.NewRecorder()
  handler(w, r)

  // Check out
  if w.Code != http.StatusOK {
    t.Fatal(w.Code, w.Body.String())
  }
}
\end{minted}

%%%%%%%%%%%%%%%%%%%%%%%%%%%%%%%%%%
\section{Working with JSON}

Let's change the input in body for our service to:

\begin{minted}[frame=single]{json}
{
  "name": "Natalia",
  "lang": "en"
}
\end{minted}

To learn how to use marshalling and unmarshalling, let's write a simple program that uses \href{https://golang.org/pkg/encoding/json/}{encoding/json package}:

\begin{minted}[frame=single]{go}
package main

import (
  "encoding/json"
  "fmt"
)

type Employee struct {
  FistName    string `json:"name"`
  LastName    string
  Internal    string `json:"-"`
  Mandatory   int    `json:"mandatory"`
  Zero        int    `json:"zero,omitempty"`
  iDoNotSeeIt int    `json:"notSeen"`
}

func main() {
  input := `{
    "name": "natalia",
    "lastName": "Buss"
  }`

  var empl Employee
  err := json.Unmarshal([]byte(input), &empl)
  if err != nil {
    // ...
    return
  }
  fmt.Println(empl.FistName)
  fmt.Println(empl.LastName)

  empl.Mandatory = 0
  empl.Zero = 0

  out, _ := json.Marshal(empl)
  fmt.Println(string(out))
}
\end{minted}

Notice: you can build your custom Marshaller/Unmarsheller. \verb|json| supports all data types.

\bigskip

Find out what \verb|json.RawMessage| is? What is a use case for it?

%%%%%%%%%%%%%%%%%%%%%%%%%%%%%%%%%%%%%%%%%%%%%%%%%%%%%%
\section{Support POST and GET with gorilla/mux}

If you want to build more complex web server, you should check \href{https://github.com/gorilla/mux}{gorilla/mux}:

\begin{minted}[frame=single]{go}
package main

import (
  "fmt"
  "log"
  "net/http"
  "github.com/gorilla/mux"
)

func HelloGetHandler(w http.ResponseWriter, r *http.Request) {
  w.WriteHeader(http.StatusOK)
  fmt.Fprintf(w, "GET")
}

func AddMsgHandler(w http.ResponseWriter, r *http.Request) {
  w.WriteHeader(http.StatusOK)
  fmt.Fprintf(w, "Post")
}

func main() {
  r := mux.NewRouter()
  r.HandleFunc("/", HelloGetHandler).Methods("GET")
  r.HandleFunc("/", AddMsgHandler).Methods("POST")

  log.Fatal(http.ListenAndServe(":8080", r))
}
\end{minted}

Refactor your application to use \verb|gorilla/mux|.

%%%%%%%%%%%%%%%%%%%%%%%%%%%%%%%%%%%%%%
\section{Web app with memory storage}

Build the following application, so we can add, display, and remove hello messages:

\begin{markdown}
- \verb|/hello_msg|, \verb|POST| - add new hello message
- \verb|/say_hello?user=natalia&lang=en|, \verb|GET| - say hello
- \verb|/hello_msg|, \verb|GET| - list all messages
- \verb|/hello_msg/{id}|, \verb|DELETE| - remove hello message
\end{markdown}

Start as a simple array, later we can build it as a map.

%%%%%%%%%%%%%%%%%%%%%%%%%%%%%%%%%%%%%%
\section{ReadTimeout and WriteTimeout for http.Server}
Remember that all IO operations should be cancel-able or timeout-able:

\begin{minted}[frame=single]{go}
srv := &http.Server{
  Addr: "8080",
  Handler: h,
  ReadTimeout: 2s,
  WriteTimeout:2s,
  MaxHeaderBytes: 1 << 20,
}

srv.ListenAndServe()
\end{minted}

%%%%%%%%%%%%%%%%%%%%%%%%%%%%%%
\pagebreak
\section{Calling remote APIs}

\begin{minted}{go}
package main

import (
  "log"
  "net/http"
  "time"
)

func main() {
  c := &http.Client{
    Timeout: 2 * time.Second,
  }

  log.Println("Fetching...")
  resp, err := c.Get(
    "https://mdn.github.io/learning-area/javascript/oojs/json/superheroes.json")

  if err != nil {
    log.Fatal(err)
  }
  defer resp.Body.Close()
}
\end{minted}

Your task is to parse the output. While looking for the best way to parse it, use your writing-tests skills, so you do not DDOS mdn.github.io.

%%%%%%%%%%%%%%%%%%%%%%%%%%%%%%
\section{Build Hero API Client}

Refactor the previous application and extract fetching list of herous to a \verb|HeroClient|:

\begin{minted}[frame=single]{go}
type HeroClient struct {
  Client *http.Client
}

func (c *HeroClient) GetThem() (string, error) {
  // your code
}

func main() {
  c := &http.Client{
    Timeout: 2 * time.Second,
  }

  hc := HeroClient{Client: c}
  
  // your code to read and display 
  // superheroes JSON
}
\end{minted}

%%%%%%%%%%%%%%%%%%%%%%%%%%%%%%
\section{Testing Calling remote APIs}

You can also test whether your calls have proper format by using \mintinline{go}{httptest}:

\begin{minted}[frame=single]{go}
package main

import (
  "fmt"
  "net/http"
  "net/http/httptest"
  "testing"

  "gotest.tools/assert"
)

func TestHeroClientAPI(t *testing.T) {

  server := httptest.NewServer(
    http.HandlerFunc(
      func(rw http.ResponseWriter, req *http.Request) {
        // Send response to be tested
        assert.Equal(t, req.URL.String(), "/some/path") 
        rw.Write([]byte(`OK`))
      }),
  )

  // Close the server when test finishes
  defer server.Close()
  // Use Client & URL from our local test server
  api := HeroClient{server.Client()}
  r, err := api.GetThem()
  assert.NilError(t, err)
  fmt.Println(r)
}
\end{minted}

Please refactor your code from previous exercise, add GET argument, and~write the test.

%%%%%%%%%%%%%%%%%%%%%%%%%%%%%%%%
\section{Working with files}

Based on https://gobyexample.com/writing-files and https://gobyexample.com/reading-files:

\begin{markdown}
1. read /etc/passwd and find a line numer with your user
2. transform passwd to json (name, pid, gid, and path) and write to \verb|${HOME}/passwd.json|
\end{markdown}

%%%%%%%%%%%%%%%%%%%%%%%%%%%%%%%%%
% https://gobyexample.com/command-line-flags
\section{Parsing CLI args}

Using an example from https://gobyexample.com/command-line-flags:

\begin{minted}[frame=single]{go}
package main

import "flag"
import "fmt"

func main() {

    wordPtr := flag.String("word", "foo", "a string")

    numbPtr := flag.Int("numb", 42, "an int")
    boolPtr := flag.Bool("fork", false, "a bool")

    var svar string
    flag.StringVar(&svar, "svar", "bar", "a string var")

    flag.Parse()

    fmt.Println("word:", *wordPtr)
    fmt.Println("numb:", *numbPtr)
    fmt.Println("fork:", *boolPtr)
    fmt.Println("svar:", svar)
    fmt.Println("tail:", flag.Args())
}
\end{minted}

build a program that prints all files or directories in a given \emph{path}. The program should let us to specify regex for the file or directories names.

\bigskip

How would you test the CLI app?

%%%%%%%%%%%%%%%%%%%%%%%%%%%%%%%%%
\section{Go Concurrency}

Let's now take a look on the built-in concurrency:

\begin{markdown}
- green threads (go routines)
- can run hundreds of thousands routines
- low overhead (dynamic stack)
- channels for communication
- scalable model
\end{markdown}

%%
%%
\subsection{Goroutines and Channels}

\begin{minted}[frame=single]{go}
package main

import (
  "fmt"
  "time"
)

func say(s string) {
  for i := 0; i < 5; i++ {
    time.Sleep(100 * time.Millisecond)
    fmt.Println(s)
  }
}

func main() {
  go say("world")
  say("hello")
}
\end{minted}

\begin{minted}[frame=single]{go}
package main

import "fmt"

func sum(s []int, c chan int) {
  sum := 0
  for _, v := range s {
    sum += v
  }
  c <- sum // send sum to c
}

func main() {
  s := []int{7, 2, 8, -9, 4, 0}

  c := make(chan int)
  go sum(s[:len(s)/2], c)
  go sum(s[len(s)/2:], c)
  x, y := <-c, <-c // receive from c

  fmt.Println(x, y, x+y)
}
\end{minted}

Based on the above examples, write a program that counts files in 2 directories. The first implementation should be sequential, the second - parallel.

%%%
\subsection{sync.Mutex}
How would you implement a global counter with sync.Mutex?

% https://tour.golang.org/concurrency/9
\begin{minted}[frame=single]{go}

type SafeCounter struct {
  numberOfFiles  int
  mux sync.Mutex
}

//c.mux.Lock()
//c.mux.Unlock()
\end{minted}

\smallskip

What are drawbacks of this solution?

\smallskip

Implement one counter with mutex and one with channel.

\subsection{Further read}

For more complex use cases:

\begin{markdown}
- https://blog.golang.org/pipelines
- https://blog.golang.org/context
\end{markdown}

We will cover the Golang concurrency in the follow-up training.

%%%%%%%%%%%%%%%%%%%%%%%%%%%%%%%%%
\section{Database Access}


\subsection{Postgres}

Package \verb|database/sql| provides generic interface for {\small SQL} databases. In our exe

\bigskip
1. Prepare the project

\begin{minted}{text}
# anywhere 
$ mkdir workshop-db
$ go mod init github.com/wojciech12/workshop-db
$ go get github.com/lib/pq
$ go get github.com/jmoiron/sqlx
\end{minted}

\bigskip
2. Run psql:

\begin{minted}{text}
# user: postgres
$ docker run --rm \
  --name workshop-psql \
  -e POSTGRES_DB=hello_world \
  -e POSTGRES_PASSWORD=nomoresecret \
  -d \
  -p 5432:5432 \
  postgres
\end{minted}

Notice:

\begin{minted}{text}
$ psql hello_world postgres -h 127.0.0.1 -p 5432
\end{minted}

\bigskip
3. Connect to db:

\begin{minted}[frame=single]{go}
package main

import (
  "database/sql"
  "fmt"
  "net/url"

  _ "github.com/lib/pq"
)

var driverName = "postgres"

func New(connectionInfo string) (*sql.DB, error) {
  db, err := sql.Open(driverName, connectionInfo)
  if err != nil {
    msg := fmt.Sprintf("cannot open db (%s) connection: %v", 
      driverName, err)
    println(msg)
    return nil, err
  }
  return db, nil
}

func main() {
  user := url.PathEscape("postgres")
  password := url.PathEscape("nomoresecret")
  host := "127.0.0.1"
  port := "5432"
  dbName := "hello_world"
  sslMode := "disable"

  connInfo := fmt.Sprintf(
    "postgres://%s:%s@%s:%s/%s?sslmode=%s",
    user, password, host, port, dbName, sslMode)

  sql, err := New(connInfo)
  if err != nil {
    panic(err)
  }
  err = sql.Ping()
  if err != nil {
    panic(err)
  }
  defer sql.Close()
}
\end{minted}

\bigskip
4. Let's create tables using the following definition:

\begin{minted}[frame=single]{go}
CREATE TABLE users (
  id BIGSERIAL PRIMARY KEY,
  first_name TEXT,
  last_name TEXT);
\end{minted}

\bigskip
5. Create table in Golang:

\begin{minted}[frame=single]{go}
func createTableIfNotExist(sql *sql.DB) {
  _, err := sql.Exec(`CREATE TABLE users (
    id  BIGSERIAL PRIMARY KEY,
    first_name TEXT,
    last_name TEXT)`)
  fmt.Printf("%v\n", err)
}
\end{minted}

\bigskip
6. Add lines:

\begin{minted}[frame=single]{go}
func insertData(sql *sql.DB, firstName string,
    lastName string) error {
  iq := `INSERT INTO users (first_name, last_name)
         VALUES ($1,$2) RETURNING id;`
  stmt, err := sql.Prepare(iq)
  if err != nil {
    return err
  }
  defer stmt.Close()
  _, err = stmt.Exec(firstName, lastName)
  if err != nil {
    return err
  }
  return nil
}
\end{minted}


\bigskip
6. Read lines:

\begin{minted}[frame=single]{go}
func readData(sql *sql.DB) error {
  s := `SELECT id, first_name, last_name FROM users`
  rows, err := sql.Query(s)
  if err != nil {
    return err
  }
  defer rows.Close()

  type person struct {
    ID         int
    FirstName  string
    SecondName string
  }

  var p person
  for rows.Next() {
    if err := rows.Scan(
      &p.ID,
      &p.FirstName,
      &p.SecondName); err != nil {
      return err
    }
    fmt.Printf("%d %s %s", p.ID, p.FirstName, p.SecondName)
  }
  return nil
\end{minted}

\bigskip
7. With sqlx\footnote{\href{https://github.com/jmoiron/sqlx}{https://github.com/jmoiron/sqlx}}, you can have more declarative code for working with your database:

\begin{minted}[frame=single]{go}
dbx := sqlx.NewDb(sql, driverName)
\end{minted}

\begin{minted}[frame=single]{go}
func insertData2(sql *sqlx.DB, firstName string,
  lastName string) error {
  type input struct {
    FirstName string `db:"first_name"`
    LastName  string `db:"last_name"`
  }
  type output struct {
    ID int64 `db:"id"`
  }

  var out output
  var in input

  in.FirstName = firstName
  in.LastName = lastName

  sqlQuery := `INSERT INTO users ( first_name,
            last_name
           ) VALUES (
         :first_name,
         :last_name) RETURNING id`

  stmt, err := sql.PrepareNamed(sqlQuery)
  if err != nil {
    return err
  }
  err = stmt.Get(&out, in)
  if err != nil {
    return err
  }
  fmt.Println(out.ID)
  return nil
}
\end{minted}

Notice: for select queries, you use \verb|Queryx| and \verb|err := rows.StructScan(&out)|.

\bigskip

8. Add support for the database in your web app.

%%
\subsection{Migrations}

Presentation of golang-migrate/migrate\footnote{\href{https://github.com/golang-migrate/migrate}{https://github.com/golang-migrate/migrate}}.

%%
\subsection{Mongodb}

A homework, prepare an application that uses mongodb as its database:

\begin{minted}{text}
$ docker run  -p 27017:27017 \
  --name da-mongo \
   -d \ 
   mongo
\end{minted}

\begin{minted}{text}
# anywhere 
$ mkdir workshop-mgo
$ go get github.com/globalsign/mgo
\end{minted}

%%%%%%%%%%%%%%%%%%%%%%%%%%%%%%%%%
\section{Logging}

Most popular: \verb|log|, \verb|sirupsen/logrus|, and, for those who want to save every CPU cycle - \verb|uber-go/zap|.

%%%%%%%%%%%%%%%%%%%%%%%%%%%%%%%%%
\section{What is more in stdlib}

With simple 25 reserve words and powerful standard library - \verb|golang.org/pkg/|

\begin{markdown}
- production ready http.Server
- cryptography (TLS, AES, RSA, HMAC, SHA, MD5)
- compression (gzip, zlib, lzw, bzip2, flate)
- filesystem modifications, subprocesses, system calls
- IO readers, writers, seekers, pipes
- SQL interface (third-party drivers needed)
- time, date
- http reverse proxy
- and more
\end{markdown}

%%%%%%%%%%%%%%%%%%%%%%%%%%%%%%%%%
\pagebreak
\section{Tools}

%%
\subsection{goreleaser}

A very sharp tool that greatly simplifies your CI/CD pipeline for Golang apps.

\begin{minted}{yaml}
project_name: myapp
release:
  github:
    owner: YOUR_USER_OR_ORG
    name: myapp
  name_template: '{{.Tag}}'
builds:
- env:
  - CGO_ENABLED=0
  goos:
  - linux
  goarch:
  - amd64
  main: .
  ldflags: -s -w -X main.version={{.Version}} -X main.commit={{.Commit}} \
    -X main.date={{.Date}}
  binary: myapp
archive:
  format: tar.gz
  name_template: '{{ .ProjectName }}_{{ .Version }}_{{ .Os }}_{{ .Arch }}{{ if .Arm
    }}v{{ .Arm }}{{ end }}'
snapshot:
  name_template: snapshot-{{.ShortCommit}}
checksum:
  name_template: '{{ .ProjectName }}_{{ .Version }}_checksums.txt'
dist: dist
dockers:
  - image: YOUR_USER_OR_ORG/myapp
\end{minted}

%%%
\subsection{Docker}

\begin{itemize}
\item Compile on your machine:\\ \verb| GOOS=linux GOARCH=amd64 CGO_ENABLED=0 go build ./...| \\ and put just binary inside the Docker
\item An alternative is to use \href{https://docs.docker.com/develop/develop-images/multistage-build/}{multi-stage Docker builds}
\item Final image \verb|alpine| or \verb|ubuntu|
\end{itemize}

%%
\subsection{Benchmarks}

Simple benchmarks with Go:

\begin{minted}[frame=single]{go}
// fib.go
func Fib(n int) int {
        if n < 2 {
                return n
        }
        return Fib(n-1) + Fib(n-2)
}

// fib_test.go
func BenchmarkFib10(b *testing.B) {
        // run the Fib function b.N times
        for n := 0; n < b.N; n++ {
                Fib(10)
        }
}
\end{minted}

\begin{minted}{text}
$ go test -bench=.  
\end{minted}

%
\subsection{Debugging}
Debugging with \verb|delve| and from vscode.

% https://github.com/golang/go/wiki/Performance
%\subsection{Profiling}

%
\subsection{Docs}

%%%%%%%%%%%%%%%%%%%%%%%%%%%%%%%%%
\section{Outlook}
What could be the next steps in learning Golang:

\begin{markdown}
1. Go Concurrency:

   - Patterns: \href{https://blog.golang.org/pipelines}{https://blog.golang.org/pipelines}
   - Context: \href{https://blog.golang.org/context}{https://blog.golang.org/contexts}

2. Graceful Shutdown, an example on \href{https://github.com/gorilla/mux#graceful-shutdown}{gorilla/mux github}
3. Running your app on Kuberentes and CloudNative
4. \href{https://github.com/wojciech12/talk_serverless_in_golang}{Serverless in Golang} on AWS and GCP
5. Observability with \href{https://github.com/wojciech12/talk_monitoring_with_prometheus}{Prometheus Stack}, Opentracing, and EFK
6. GPRC ?
7. Golang Dev for K8S

Contact: \email{wbarczynski.pro@gmail.com}
\end{markdown}

\section{References}

\begin{markdown}
- https://github.com/golang/go/wiki/CodeReviewComments
- https://golang.com/doc/effective_go.html
- http://devs.cloudimmunity.com/gotchas-and-common-mistakes-in-go-golang
\end{markdown}

\end{document}